\documentclass[leqno, 11pt]{report}

% Formatting and graphics
\usepackage{parskip}
\usepackage{graphicx}
\graphicspath{ {./images/} }

% Fonts
\usepackage{fontspec}
% \setmainfont{Gentium Plus}
\newfontfamily\greekfont[Script=Greek]{Gentium Plus}
\newfontfamily\greekfontsf[Script=Greek]{Gentium Plus}

% Multi-language support
\usepackage{polyglossia}
\usepackage{csquotes}
\setdefaultlanguage{english}
\setotherlanguage[variant=ancient]{greek}

% Bibliography
\usepackage[backend=biber, bibencoding=utf8]{biblatex}
\addbibresource{bibliography.bib}

% ams
\usepackage{mathtools}
\usepackage{amsmath}
\usepackage{amsfonts}
\usepackage{amssymb}

% Formatting and custom environments
\usepackage[printsolution=true]{exercises}
\usepackage{minted}

% Counters
\usepackage{chngcntr}

\counterwithin*{equation}{section}
\counterwithin*{equation}{subsection}

\newcounter{homework_problem_counter}
\newenvironment{homework}[1][]
  {\refstepcounter{homework_problem_counter}\par#1\par\nobreak}
  {\par}

\counterwithin*{equation}{homework_problem_counter}

%pgfplots
\usepackage{pgfplots}
\pgfplotsset{compat=newest}
\usepgfplotslibrary{statistics}
\usepgfplotslibrary{fillbetween}

%colours
\usepackage{xcolor}

\usepackage{nicefrac}

%tikzpicture
\usepackage{tikz}
\usepackage{scalerel}
\usepackage{pict2e}
\usepackage{tkz-euclide}
\usetikzlibrary{calc}
\usetikzlibrary{patterns,arrows.meta}
\usetikzlibrary{shadows}
\usetikzlibrary{external}

\pgfplotsset{
  standard/.style={
    axis line style = thick,
    trig format=rad,
    enlargelimits,
    axis x line*=middle,
    axis y line*=middle,
    enlarge x limits=0.15,
    enlarge y limits=0.15,
    every axis x label/.style={at={(current axis.right of origin)},anchor=north west},
    every axis y label/.style={at={(current axis.above origin)},anchor=south east},
    grid=both,
    ticklabel style={font=\tiny}
  }
}

\title{Notes on\\\textit{Calculus, Early Transcendentals}}
\author{Jonathan Llovet}
\date{\today}

\begin{document}
\maketitle

\tableofcontents

\newpage

\setcounter{chapter}{-1}

\chapter{Opening Remarks}

\date{May 17, 2022}

I am reading \textit{Calculus, Early Transcendentals} by \citeauthor{briggsCalculusEarlyTranscendentals2019}
as part of my Bachelor's of Science at Indiana University East. The following are my reflections, notes, and
exercises for Calculus 1, Math-215.

Preparing to study calculus again,
I think back to the readings I did previously
at St. John's of Leibniz and Newton, their predecessors, and their followers.
Though the spirit of \citetitle{briggsCalculusEarlyTranscendentals2019}
is different than those books, I will be keeping their various views in mind,
returning to them when I can to compare their treatments of various topics with those
of these authors.

I'm curious to see how my attitudes towards mathematics and the relation it has
to natural philosophy will have changed in light of my work in computer science
and algorithmics in the time since I left St. John's. At the outset, I know that
my outlook and approaches have become more practice-minded, in part out of necessity,
to meet the demands of my work, and in part to allow me to act quickly on ideas
that I am developing.

While this has been fruitful in many ways, I have been growing concerned that
that practical-mindedness has sometimes been at the expense of my circumspection
about the nature and details of the things that I have been studying.
In part to address that set of concerns, and in part to approach mathematics intentionally
with dedicated focus, so that I can pursue my other interests that rely heavily
on mathematics, where I can be at liberty to make my own estimations of how things work,
rather than to rely on the presentations of others, I'm enrolling here.
In essence, though this is a technical program, where I will be pursuing
goals of technical proficiency in the various branches of mathematics I'm studying,
I am here as well to re- take up the path of my education in the liberal arts.

A reminder for reflection:

\begin{quote}
    La filosofia è scritta in questo grandissimo
    libro che continuamente ci sta aperto innanzi a gli occhi (io dico l'universo),
    ma non si può intendere se prima non s'impara a intender la lingua,
    e conoscer i caratteri, ne' quali è scritto.
    Egli è scritto in lingua matematica,
    e i caratteri son triangoli, cerchi, ed altre figure geometriche,
    senza i quali mezi è impossibile a intenderne umanamente parola;
    senza questi è un aggirarsi vanamente per un oscuro laberinto.

    - \citeauthor{galileogalileiSaggiatore}
\end{quote}

\chapter{Functions}

\section{Review of Functions}

\newpage
\subsection{Online Exercises}
\setcounter{exercises@exercisenumber}{1}
\setcounter{exercises@solutionnumber}{1}

\begin{exercise}
    \textbf{Given:}
    \begin{itemize}
        \item A cylindrical water tower with a radius of $15m$ and a height of $60m$ is filled to a height of $h$.
        \item The volume $V$ of water​ (in cubic​ meters) is given by the function​ $g(h) = 225\pi h$.
    \end{itemize}

    \textbf{Find:}
    \begin{itemize}
        \item The appropriate domain of the function, the independent and dependent variables
    \end{itemize}
\end{exercise}
\rule{\textwidth}{0.4pt}

\begin{solution}
    While the function $g(h) = 225\pi h$ would accept any value from $\{ h \mid h \in \mathbb{R}\}$ in general,
    in the context of this question, we are only concerned with the values of $h$ that correspond to possible
    heights of water that can be put into the tower.
    
    Because there cannot be a negative amount of water in the tower, we know that $h \geq 0$.
    Similarly, because the water tower itself is $60m$ tall, the amount of water that can be put in the tower
    is limited to $60m$, meaning that $h \leq 60$.
    Taking these restrictions together: $$0m \leq h \leq 60m$$

    Finally, since $h$ can vary freely within the domain and is not determined by another quantity in the function $g(h)$,

    and since $V$ is determined by the function $g(h)$,
    
    therefore, for the equation $g(h) = 225\pi h$:

    \fbox{\parbox{\textwidth}{
        \begin{itemize}
            \item The domain is $[0, 60]$
            \item The independent variable is $h$
            \item The dependent variable is $V$
        \end{itemize}
        }
    }

    \hfill ANS $\square$
\end{solution}

\newpage
\begin{exercise}
    \textbf{Given:}
    \begin{itemize}
        \item A cylindrical water tower with a radius of $13m$ and a height of $60m$ is filled to a height of $h$.
        \item The volume $V$ of water​ (in cubic​ meters) is given by the function​ $g(h) = 169\pi h$.
    \end{itemize}

    \textbf{Find:}
    \begin{itemize}
        \item The appropriate domain of the function, the independent and dependent variables
    \end{itemize}
\end{exercise}
\rule{\textwidth}{0.4pt}

\begin{solution}
    While the function $g(h) = 169\pi h$ would accept any value from $\{ h \mid h \in \mathbb{R}\}$ in general,
    in the context of this question, we are only concerned with the values of $h$ that correspond to possible
    heights of water that can be put into the tower.
    
    Because there cannot be a negative amount of water in the tower, we know that $h \geq 0$.
    Similarly, because the water tower itself is $60m$ tall, the amount of water that can be put in the tower
    is limited to $60m$, meaning that $h \leq 60$.
    Taking these restrictions together: $$0m \leq h \leq 60m$$

    Finally, since $h$ can vary freely within the domain and is not determined by another quantity in the function $g(h)$,

    and since $V$ is determined by the function $g(h)$,
    
    therefore, for the equation $g(h) = 169\pi h$:
    
    \fbox{\parbox{\textwidth}{
        \begin{itemize}
            \item The domain is $[0, 60]$
            \item The independent variable is $h$
            \item The dependent variable is $V$
        \end{itemize}
        }
    }

    \hfill ANS $\square$
\end{solution}

\begin{homework}
    \newpage
    \begin{exercise}
        \renewcommand{\theequation}{\arabic{equation}}
        \textbf{Given:}
        \begin{itemize}
            \item $f(x) = 3x^2 - 7x + 2$
        \end{itemize}

        \textbf{Find:}
        \begin{itemize}
            \item Simplify the difference quotient $ \frac{f(x+h)-f(x)}{h}$ for $f(x)$.
        \end{itemize}
    \end{exercise}
    \rule{\textwidth}{0.4pt}

    \begin{solution}
        In order to simplify the difference quotient,
        we must first find $f(x+h)$,
        so that we can substitute it in.

        \begin{align}
            f(x) &= 3x^2 - 7x + 2\\
            f(x+h) &= 3(x+h)^2 - 7(x+h) + 2
                &&\text{(by substitution)} \notag\\
            &= 3(x^2 + 2xh + h^2) - 7(x+h) + 2
                &&\text{(by expansion)} \notag\\
            &= 3x^2 + 6xh + 3h^2 - 7x - 7h + 2
                &&\text{(by distribution)} \notag\\
            &= 3x^2 + 6xh + 3h^2 - 7x - 7h + 2
                &&\text{(by distribution)}
        \end{align}

        Now that we have a value for $f(x+h)$,
        we can substitute the value for it into the difference quotient.

        \begin{align}
            q &= \frac{f(x+h)-f(x)}{h} \notag\\
            &= \frac{( 3x^2 + 6xh + 3h^2 - 7x - 7h + 2 ) - (3x^2 - 7x + 2)}{h}\notag
                &&\text{(by substitution)} \notag\\
            &= \frac{ 6xh + 3h^2 - 7h }{h}\notag
                &&\text{(by simplification)} \notag\\
            &= \frac{ h(6x + 3h - 7) }{h}\notag
                &&\text{(by factoring)} \notag\\
            &= 6x + 3h - 7 \notag
                &&\text{(by simplification)} \notag
        \end{align}

        \fbox{\parbox{\textwidth}{
            $\therefore$ The difference quotient for $f(x)$ is $6x + 3h - 7$
            }
        }

        \hfill ANS $\square$
    \end{solution}

\end{homework}

\newpage
\section{Representing Functions}

\subsection{Synthetic Division}
    Synthetic division is used to determine whether a zero of a function
    is a factor of a polynomial.
    Core algorithm:
    \begin{itemize}
        \item Start with the leading coefficient.
        \item Multiply the zero by that coefficient to produce an intermediate term. 
        \item Add that intermediate term to the next coefficient to produce a remainder.
        \item Multiply the zero by that remainder to produce the next intermediate term.
        \item Repeat until the final coefficient is reached.
        \item If the final remainder $= 0$, then the zero is a factor of the polynomial.
        \item If the final remainder $\neq 0$, then the zero is not a factor of the polynomial.
    \end{itemize}
    \newpage
    \inputminted[linenos, breaklines]{python}{001/section_2/synthetic_division.py}
    \newpage

\subsection{Graphics}
    In order to be able to work with graphing images, I am using pgfplots. Here is an example.

    \begin{tikzpicture}
        \begin{axis}[standard,
            width=10cm,
            xtick={-10, -8, -6, -4, -2, 0, 2, 4, 6, 8, 10},
            ytick={-10, -8, -6, -4, -2, 0, 2, 4, 6, 8, 10},
            samples=1000,
            xlabel={$x$},
            ylabel={$y$},
            xmin=-10,xmax=10,
            ymin=-10,ymax=10
        ]

            \addplot[line width=0.4mm, red, name path=F,domain={-10:10}]{x^2};
        \end{axis}
    \end{tikzpicture}




\printbibliography

\end{document}
