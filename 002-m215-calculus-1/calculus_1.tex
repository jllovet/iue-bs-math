\documentclass[leqno, 11pt]{report}

\usepackage{parskip}
\usepackage{graphicx}
\graphicspath{ {./images/} }

\usepackage{fontspec}
\setmainfont{Gentium Plus}
\newfontfamily\greekfont[Script=Greek]{Gentium Plus}
\newfontfamily\greekfontsf[Script=Greek]{Gentium Plus}

\usepackage{polyglossia}
\usepackage{csquotes}
\setdefaultlanguage{english}
\setotherlanguage[variant=ancient]{greek}

\usepackage[backend=biber, bibencoding=utf8]{biblatex}
\addbibresource{bibliography.bib}

\usepackage{amsmath}
\usepackage{amsfonts}
\usepackage{amssymb}
\usepackage[printsolution=true]{exercises}

\numberwithin{equation}{section}

\title{Notes on\\\textit{Calculus, Early Transcendentals}}
\author{Jonathan Llovet}
\date{May 17, 2022}

\begin{document}
\maketitle

I am reading \textit{Calculus, Early Transcendentals} by \citeauthor{briggsCalculusEarlyTranscendentals2019}
as part of my Bachelor's of Science at Indiana University East. The following are my reflections, notes, and
exercises for Calculus 1, Math-215.

Preparing to study calculus again,
I think back to the readings I did previously
at St. John's of Leibniz and Newton, their predecessors, and their followers.
Though the spirit of \citetitle{briggsCalculusEarlyTranscendentals2019}
is different than those books, I will be keeping their various views in mind,
returning to them when I can to compare their treatments of various topics with those
of these authors.

I'm curious to see how my attitudes towards mathematics and the relation it has
to natural philosophy will have changed in light of my work in computer science
and algorithmics in the time since I left St. John's. At the outset, I know that
my outlook and approaches have become more practice-minded, in part out of necessity,
to meet the demands of my work, and in part to allow me to act quickly on ideas
that I am developing.

While this has been fruitful in many ways, I have been growing concerned that
that practical-mindedness has sometimes been at the expense of my circumspection
about the nature and details of the things that I have been studying.
In part to address that set of concerns, and in part to approach mathematics intentionally
with dedicated focus, so that I can pursue my other interests that rely heavily
on mathematics, where I can be at liberty to make my own estimations of how things work,
rather than to rely on the presentations of others, I'm enrolling here.
In essence, though this is a technical program, where I will be pursuing
goals of technical proficiency in the various branches of mathematics I'm studying,
I am here as well to re- take up the path of my education in the liberal arts.

\newpage
A reminder for reflection:

\begin{quote}
    La filosofia è scritta in questo grandissimo
    libro che continuamente ci sta aperto innanzi a gli occhi (io dico l'universo),
    ma non si può intendere se prima non s'impara a intender la lingua,
    e conoscer i caratteri, ne' quali è scritto.
    Egli è scritto in lingua matematica,
    e i caratteri son triangoli, cerchi, ed altre figure geometriche,
    senza i quali mezi è impossibile a intenderne umanamente parola;
    senza questi è un aggirarsi vanamente per un oscuro laberinto.

    - \citeauthor{galileogalileiSaggiatore}
\end{quote}

\newpage

\tableofcontents

\newpage
\chapter{Functions}

\section{Review of Functions}

\newpage
\subsection{Online Exercises}
\setcounter{exercises@exercisenumber}{1}
\setcounter{exercises@solutionnumber}{1}

\input{001/section_1/subsection_1_hw_2022_05_17}
\input{001/section_1/subsection_1_hw_2022_05_18}
\input{001/section_1/subsection_1_hw_2022_05_19}



\printbibliography

\end{document}
