\begin{homework}
    \newpage
    \date{May 19, 2022}

    \begin{exercise}
        \textbf{Given:}
        \begin{itemize}
            \item The graph of $ y=x^2 $​
        \end{itemize}

        \textbf{Find:}
        \begin{itemize}
            \item The process for obtaining the graph ​$ 7(x+9)^2 + 2 $
        \end{itemize}
    \end{exercise}
    \rule{\textwidth}{0.4pt}

    \begin{solution}
        We begin with the original graph $ y=x^2 $

        \begin{tikzpicture}
            \begin{axis}[standard,
                width=10cm,
                xtick={-10, -8, -6, -4, -2, 0, 2, 4, 6, 8, 10},
                ytick={-10, -8, -6, -4, -2, 0, 2, 4, 6, 8, 10},
                samples=1000,
                xlabel={$x$},
                ylabel={$y$},
                xmin=-10,xmax=10,
                ymin=-10,ymax=10
            ]
        
            \addplot[line width=0.4mm, red, domain={-10:10}]{x^2};
            \end{axis}
        \end{tikzpicture}

        Next, we shift the graph to the left by 9 units.
        At this point, we have a graph of $ y=(x+9)^2 $.
        Note the change in the units shown on the graph.
        
        \nopagebreak
        
        \begin{tikzpicture}
            \begin{axis}[standard,
                width=10cm,
                xtick={-20, -18, -16, -14, -12, -10, -8, -6, -4, -2, 0, 2},
                ytick={-10, -8, -6, -4, -2, 0, 2, 4, 6, 8, 10, 12, 14, 16, 18, 20},
                samples=1000,
                xlabel={$x$},
                ylabel={$y$},
                xmin=-20,xmax=2,
                ymin=-10,ymax=10
            ]
        
            \addplot[line width=0.4mm, red, domain={-20:2}]{(x+9)^2};
            \end{axis}
        \end{tikzpicture}

        After shifting to the left, we scale the graph by a factor of 7.
        Because of the shape of our graph, this results in a narrowing or stretching.
        At this point, we have a graph of $ y=7(x+9)^2 $.
        
        \nopagebreak
        
        \begin{tikzpicture}
            \begin{axis}[standard,
                width=10cm,
                xtick={-20, -18, -16, -14, -12, -10, -8, -6, -4, -2, 0, 2},
                ytick={-10, -8, -6, -4, -2, 0, 2, 4, 6, 8, 10, 12, 14, 16, 18, 20},
                samples=1000,
                xlabel={$x$},
                ylabel={$y$},
                xmin=-20,xmax=2,
                ymin=-10,ymax=10
            ]
        
            \addplot[line width=0.4mm, red, domain={-20:2}]{7*(x+9)^2};
            \end{axis}
        \end{tikzpicture}

        After scaling the graph, next we shift the graph up 2 units.
        At this point, we have a graph of $ y=7(x+9)^2 + 2$.
        
        \nopagebreak

        \begin{tikzpicture}
            \begin{axis}[standard,
                width=10cm,
                xtick={-20, -18, -16, -14, -12, -10, -8, -6, -4, -2, 0, 2},
                ytick={-10, -8, -6, -4, -2, 0, 2, 4, 6, 8, 10, 12, 14, 16, 18, 20},
                samples=1000,
                xlabel={$x$},
                ylabel={$y$},
                xmin=-20,xmax=2,
                ymin=-10,ymax=10
            ]
        
            \addplot[line width=0.4mm, red, domain={-20:2}]{7*(x+9)^2 + 2};
            \end{axis}
        \end{tikzpicture}

        \fbox{\parbox{\textwidth}{
            In summary, the transformations we have to make on the graph are:
            \begin{itemize}
                \item Shift to the left 9 units
                \item Scale by a factor of 7
                \item Shift up by 2 units
            \end{itemize}
            }
        }
        \nopagebreak
        \hfill ANS $\square$
    \end{solution}
\end{homework}