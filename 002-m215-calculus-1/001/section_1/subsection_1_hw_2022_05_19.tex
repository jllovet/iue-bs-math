\begin{homework}
    \newpage
    \date{May 19, 2022}

    \begin{exercise}
        \textbf{Given:}
        \begin{itemize}
            \item The graph of $ y=x^2 $​
        \end{itemize}

        \textbf{Find:}
        \begin{itemize}
            \item The process for obtaining the graph ​$ 7(x+9)^2 + 2 $
        \end{itemize}
    \end{exercise}
    \rule{\textwidth}{0.4pt}

    \begin{solution}
        We begin with the original graph $ y=x^2 $

        \begin{tikzpicture}
            \begin{axis}[standard,
                width=10cm,
                xtick={-10, -8, -6, -4, -2, 0, 2, 4, 6, 8, 10},
                ytick={-10, -8, -6, -4, -2, 0, 2, 4, 6, 8, 10},
                samples=1000,
                xlabel={$x$},
                ylabel={$y$},
                xmin=-10,xmax=10,
                ymin=-10,ymax=10
            ]
        
            \addplot[line width=0.4mm, red, domain={-10:10}]{x^2};
            \end{axis}
        \end{tikzpicture}

        Next, we shift the graph to the left by 9 units.
        At this point, we have a graph of $ y=(x+9)^2 $.
        Note the change in the units shown on the graph.
        
        \nopagebreak
        
        \begin{tikzpicture}
            \begin{axis}[standard,
                width=10cm,
                xtick={-20, -18, -16, -14, -12, -10, -8, -6, -4, -2, 0, 2},
                ytick={-10, -8, -6, -4, -2, 0, 2, 4, 6, 8, 10, 12, 14, 16, 18, 20},
                samples=1000,
                xlabel={$x$},
                ylabel={$y$},
                xmin=-20,xmax=2,
                ymin=-10,ymax=10
            ]
        
            \addplot[line width=0.4mm, red, domain={-20:2}]{(x+9)^2};
            \end{axis}
        \end{tikzpicture}

        After shifting to the left, we scale the graph by a factor of 7.
        Because of the shape of our graph, this results in a narrowing or stretching.
        At this point, we have a graph of $ y=7(x+9)^2 $.
        
        \nopagebreak
        
        \begin{tikzpicture}
            \begin{axis}[standard,
                width=10cm,
                xtick={-20, -18, -16, -14, -12, -10, -8, -6, -4, -2, 0, 2},
                ytick={-10, -8, -6, -4, -2, 0, 2, 4, 6, 8, 10, 12, 14, 16, 18, 20},
                samples=1000,
                xlabel={$x$},
                ylabel={$y$},
                xmin=-20,xmax=2,
                ymin=-10,ymax=10
            ]
        
            \addplot[line width=0.4mm, red, domain={-20:2}]{7*(x+9)^2};
            \end{axis}
        \end{tikzpicture}

        After scaling the graph, next we shift the graph up 2 units.
        At this point, we have a graph of $ y=7(x+9)^2 + 2$.
        
        \nopagebreak

        \begin{tikzpicture}
            \begin{axis}[standard,
                width=10cm,
                xtick={-20, -18, -16, -14, -12, -10, -8, -6, -4, -2, 0, 2},
                ytick={-10, -8, -6, -4, -2, 0, 2, 4, 6, 8, 10, 12, 14, 16, 18, 20},
                samples=1000,
                xlabel={$x$},
                ylabel={$y$},
                xmin=-20,xmax=2,
                ymin=-10,ymax=10
            ]
        
            \addplot[line width=0.4mm, red, domain={-20:2}]{7*(x+9)^2 + 2};
            \end{axis}
        \end{tikzpicture}

        \fbox{\parbox{\textwidth}{
            In summary, the transformations we have to make on the graph are:
            \begin{itemize}
                \item Shift to the left 9 units
                \item Scale by a factor of 7
                \item Shift up by 2 units
            \end{itemize}
            }
        }
        \nopagebreak
        \hfill ANS $\square$
    \end{solution}
\end{homework}


\begin{homework}
    \newpage
    \date{May 19, 2022}

    \begin{exercise}
        \textbf{Given:}
        \begin{itemize}
            \item Two pairs of temperatures in Fahrenheit and Celsius $ (F, C) $:
            \begin{itemize}
                \item $ (212, 100) $
                \item $ (32, 0) $
            \end{itemize}
        \end{itemize}

        \textbf{Find:}
        \begin{itemize}
            \item The linear equation represented by the function $ C = c(F) $
            that returns the corresponding value in Celsius when given a temperature in Fahrenheit.
            \item The temperature where $ C = F $
        \end{itemize}
    \end{exercise}
    \rule{\textwidth}{0.4pt}

    \begin{solution}
        
        Since we have two pairs of temperatures, we can treat these as points of our function $ c(F) $.
        Using this, we can calculate the slope as follows:

        \begin{align*}
            m &= \frac{C_2 - C_1}{F_2 - F_1}
                &&\text{(definition)}\\
              &= \frac{100-0}{212-32}
                &&\text{(substitution)}\\
              &= \frac{100}{180}
                &&\text{(simplifying)}\\
            &= \frac{5}{9}
                &&\text{(simplifying)}
        \end{align*}
        
        Since we know the slope $m$ of $c(F)$, but we do not know the $y$-intercept $b$,
        we can use the \textit{Point-Slope Form} of the linear equation $c(F)$ to find the rest of the function.
        
        Using the pair of temperature values $ (212, 100) $:
        \begin{align*}
            C - C_1 &= m(F - F_1)
                &&\text{(definition)}\\
            C - 100 &= m(F - 212)
                &&\text{(substitution)}\\
            C &= \frac{5}{9}(F - 212) + 100
                &&\text{(adding $100$)}\\
              &= \frac{5}{9}F - \frac{5}{9}(212) + 100
                &&\text{(factoring)}\\
              &= \frac{5}{9}F - \frac{1060}{9} + 100
                &&\text{(simplifying)}\\
              &= \frac{5}{9}F - \frac{1060}{9} + \frac{900}{9}
                &&\text{(changing denominator)}\\
              &= \frac{5}{9}F - \frac{160}{9}
                &&\text{(simplifying)}
        \end{align*}

        Now we have our linear equation for converting temperatures in Fahrenheit to ones in Celsius.
        Graphing our equation, we can see that at -40, the values are equal.

        \begin{tikzpicture}
            \begin{axis}[standard,
                title={$ C = \frac{5}{9}F - \frac{160}{9} $},
                width=10cm,
                xtick distance=40,
                ytick distance=40,
                samples=2000,
                xlabel={$F$},
                ylabel={$C$},
                xmin=-100,xmax=250,
                ymin=-100,ymax=200,
            ]
        
            \addplot[line width=0.4mm, red, domain={-120:250}]{5/9*x - 160/9} node[right,pos=1] {$g(x)$};;
            \end{axis}
        \end{tikzpicture}

        Demonstrating this algebraically:
        \begin{align*}
            C &= F\\
            C &= \frac{5}{9} C - \frac{160}{9}
                &&\text{(substitution)}\\
            C - \frac{5}{9} C &= - \frac{160}{9}
                &&\text{(subtracting)}\\
            C (1 - \frac{5}{9}) &= - \frac{160}{9}
                &&\text{(factoring)}\\
            C &= - \left( \frac{160}{9} \right)\frac{1}{4/9}
                &&\text{(dividing)}\\
              &= - \left( \frac{160}{9} \right)\frac{9}{4}
                &&\text{(taking reciprocal)}\\
              &= - \frac{160}{4}
                &&\text{(simplifying)}\\
              &= - 40
                &&\text{(simplifying)}
        \end{align*}
        
        \fbox{\parbox{\textwidth}{
            Therefore,
            \begin{itemize}
                \item $c(F) = C = \frac{5}{9}F - \frac{160}{9}$
                \item At $-40$, $ F = C $
            \end{itemize}
            }
        }
        \nopagebreak
        \hfill ANS $\square$
    \end{solution}
\end{homework}
