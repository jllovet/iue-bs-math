\newpage
\subsection{Online Exercises}
\setcounter{exercises@exercisenumber}{1}
\setcounter{exercises@solutionnumber}{1}

\begin{homework}
    \date{May 17, 2022}

    \begin{exercise}
        \textbf{Given:}
        \begin{itemize}
            \item A cylindrical water tower with a radius of $15m$ and a height of $60m$ is filled to a height of $h$.
            \item The volume $V$ of water​ (in cubic​ meters) is given by the function​ $g(h) = 225\pi h$.
        \end{itemize}
    
        \textbf{Find:}
        \begin{itemize}
            \item The appropriate domain of the function, the independent and dependent variables
        \end{itemize}
    \end{exercise}
    \rule{\textwidth}{0.4pt}
    
    \begin{solution}
        While the function $g(h) = 225\pi h$ would accept any value from $\{ h \mid h \in \mathbb{R}\}$ in general,
        in the context of this question, we are only concerned with the values of $h$ that correspond to possible
        heights of water that can be put into the tower.
        
        Because there cannot be a negative amount of water in the tower, we know that $h \geq 0$.
        Similarly, because the water tower itself is $60m$ tall, the amount of water that can be put in the tower
        is limited to $60m$, meaning that $h \leq 60$.
        Taking these restrictions together: $$0m \leq h \leq 60m$$
    
        Finally, since $h$ can vary freely within the domain and is not determined by another quantity in the function $g(h)$,
    
        and since $V$ is determined by the function $g(h)$,
        
        therefore, for the equation $g(h) = 225\pi h$:
    
        \fbox{\parbox{\textwidth}{
            \begin{itemize}
                \item The domain is $[0, 60]$
                \item The independent variable is $h$
                \item The dependent variable is $V$
            \end{itemize}
            }
        }
        \nopagebreak
        \hfill ANS $\square$
    \end{solution}
\end{homework}

\begin{homework}
    \newpage
    \date{May 17, 2022}

    \begin{exercise}
        \textbf{Given:}
        \begin{itemize}
            \item A cylindrical water tower with a radius of $13m$ and a height of $60m$ is filled to a height of $h$.
            \item The volume $V$ of water​ (in cubic​ meters) is given by the function​ $g(h) = 169\pi h$.
        \end{itemize}

        \textbf{Find:}
        \begin{itemize}
            \item The appropriate domain of the function, the independent and dependent variables
        \end{itemize}
    \end{exercise}
    \rule{\textwidth}{0.4pt}

    \begin{solution}
        While the function $g(h) = 169\pi h$ would accept any value from $\{ h \mid h \in \mathbb{R}\}$ in general,
        in the context of this question, we are only concerned with the values of $h$ that correspond to possible
        heights of water that can be put into the tower.
        
        Because there cannot be a negative amount of water in the tower, we know that $h \geq 0$.
        Similarly, because the water tower itself is $60m$ tall, the amount of water that can be put in the tower
        is limited to $60m$, meaning that $h \leq 60$.
        Taking these restrictions together: $$0m \leq h \leq 60m$$

        Finally, since $h$ can vary freely within the domain and is not determined by another quantity in the function $g(h)$,

        and since $V$ is determined by the function $g(h)$,
        
        therefore, for the equation $g(h) = 169\pi h$:
        
        \fbox{\parbox{\textwidth}{
            \begin{itemize}
                \item The domain is $[0, 60]$
                \item The independent variable is $h$
                \item The dependent variable is $V$
            \end{itemize}
            }
        }
        \nopagebreak
        \hfill ANS $\square$
    \end{solution}
\end{homework}

\begin{homework}
    \newpage
    \date{May 18, 2022}

    \begin{exercise}
        \renewcommand{\theequation}{\arabic{equation}}
        \textbf{Given:}
        \begin{itemize}
            \item $f(x) = 3x^2 - 7x + 2$
        \end{itemize}

        \textbf{Find:}
        \begin{itemize}
            \item Simplify the difference quotient $ \frac{f(x+h)-f(x)}{h}$ for $f(x)$.
        \end{itemize}
    \end{exercise}
    \rule{\textwidth}{0.4pt}

    \begin{solution}
        In order to simplify the difference quotient,
        we must first find $f(x+h)$,
        so that we can substitute it in.

        \begin{align}
            f(x) &= 3x^2 - 7x + 2\\
            f(x+h) &= 3(x+h)^2 - 7(x+h) + 2
                &&\text{(by substitution)} \notag\\
            &= 3(x^2 + 2xh + h^2) - 7(x+h) + 2
                &&\text{(by expansion)} \notag\\
            &= 3x^2 + 6xh + 3h^2 - 7x - 7h + 2
                &&\text{(by distribution)} \notag\\
            &= 3x^2 + 6xh + 3h^2 - 7x - 7h + 2
                &&\text{(by distribution)}
        \end{align}

        Now that we have a value for $f(x+h)$,
        we can substitute the value for it into the difference quotient.

        \begin{align}
            q &= \frac{f(x+h)-f(x)}{h} \notag\\
            &= \frac{( 3x^2 + 6xh + 3h^2 - 7x - 7h + 2 ) - (3x^2 - 7x + 2)}{h}\notag
                &&\text{(by substitution)} \notag\\
            &= \frac{ 6xh + 3h^2 - 7h }{h}\notag
                &&\text{(by simplification)} \notag\\
            &= \frac{ h(6x + 3h - 7) }{h}\notag
                &&\text{(by factoring)} \notag\\
            &= 6x + 3h - 7 \notag
                &&\text{(by simplification)} \notag
        \end{align}

        \fbox{\parbox{\textwidth}{
            $\therefore$ The difference quotient for $f(x)$ is $6x + 3h - 7$
            }
        }
        \nopagebreak
        \hfill ANS $\square$
    \end{solution}

\end{homework}


\begin{homework}
    \newpage
    \date{May 19, 2022}

    \begin{exercise}
        \textbf{Given:}
        \begin{itemize}
            \item The graph of $ y=x^2 $​
        \end{itemize}

        \textbf{Find:}
        \begin{itemize}
            \item The process for obtaining the graph ​$ 7(x+9)^2 + 2 $
        \end{itemize}
    \end{exercise}
    \rule{\textwidth}{0.4pt}

    \begin{solution}
        We begin with the original graph $ y=x^2 $

        \begin{tikzpicture}
            \begin{axis}[standard,
                width=10cm,
                xtick={-10, -8, -6, -4, -2, 0, 2, 4, 6, 8, 10},
                ytick={-10, -8, -6, -4, -2, 0, 2, 4, 6, 8, 10},
                samples=1000,
                xlabel={$x$},
                ylabel={$y$},
                xmin=-10,xmax=10,
                ymin=-10,ymax=10
            ]
        
            \addplot[line width=0.4mm, red, domain={-10:10}]{x^2};
            \end{axis}
        \end{tikzpicture}

        Next, we shift the graph to the left by 9 units.
        At this point, we have a graph of $ y=(x+9)^2 $.
        Note the change in the units shown on the graph.
        
        \nopagebreak
        
        \begin{tikzpicture}
            \begin{axis}[standard,
                width=10cm,
                xtick={-20, -18, -16, -14, -12, -10, -8, -6, -4, -2, 0, 2},
                ytick={-10, -8, -6, -4, -2, 0, 2, 4, 6, 8, 10, 12, 14, 16, 18, 20},
                samples=1000,
                xlabel={$x$},
                ylabel={$y$},
                xmin=-20,xmax=2,
                ymin=-10,ymax=10
            ]
        
            \addplot[line width=0.4mm, red, domain={-20:2}]{(x+9)^2};
            \end{axis}
        \end{tikzpicture}

        After shifting to the left, we scale the graph by a factor of 7.
        Because of the shape of our graph, this results in a narrowing or stretching.
        At this point, we have a graph of $ y=7(x+9)^2 $.
        
        \nopagebreak
        
        \begin{tikzpicture}
            \begin{axis}[standard,
                width=10cm,
                xtick={-20, -18, -16, -14, -12, -10, -8, -6, -4, -2, 0, 2},
                ytick={-10, -8, -6, -4, -2, 0, 2, 4, 6, 8, 10, 12, 14, 16, 18, 20},
                samples=1000,
                xlabel={$x$},
                ylabel={$y$},
                xmin=-20,xmax=2,
                ymin=-10,ymax=10
            ]
        
            \addplot[line width=0.4mm, red, domain={-20:2}]{7*(x+9)^2};
            \end{axis}
        \end{tikzpicture}

        After scaling the graph, next we shift the graph up 2 units.
        At this point, we have a graph of $ y=7(x+9)^2 + 2$.
        
        \nopagebreak

        \begin{tikzpicture}
            \begin{axis}[standard,
                width=10cm,
                xtick={-20, -18, -16, -14, -12, -10, -8, -6, -4, -2, 0, 2},
                ytick={-10, -8, -6, -4, -2, 0, 2, 4, 6, 8, 10, 12, 14, 16, 18, 20},
                samples=1000,
                xlabel={$x$},
                ylabel={$y$},
                xmin=-20,xmax=2,
                ymin=-10,ymax=10
            ]
        
            \addplot[line width=0.4mm, red, domain={-20:2}]{7*(x+9)^2 + 2};
            \end{axis}
        \end{tikzpicture}

        \fbox{\parbox{\textwidth}{
            In summary, the transformations we have to make on the graph are:
            \begin{itemize}
                \item Shift to the left 9 units
                \item Scale by a factor of 7
                \item Shift up by 2 units
            \end{itemize}
            }
        }
        \nopagebreak
        \hfill ANS $\square$
    \end{solution}
\end{homework}