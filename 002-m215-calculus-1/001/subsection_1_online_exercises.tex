\newpage
\subsection{Online Exercises}
\setcounter{exercises@exercisenumber}{1}
\setcounter{exercises@solutionnumber}{1}

\begin{exercise}
    \textbf{Given:}
    \begin{itemize}
        \item A cylindrical water tower with a radius of $15m$ and a height of $60m$ is filled to a height of $h$.
        \item The volume $V$ of water​ (in cubic​ meters) is given by the function​ $g(h) = 225\pi h$.
    \end{itemize}

    \textbf{Find:}
    \begin{itemize}
        \item The appropriate domain of the function, the independent and dependent variables
    \end{itemize}
\end{exercise}
\rule{\textwidth}{0.4pt}

\begin{solution}
    While the function $g(h) = 225\pi h$ would accept any value from $\{ h \mid h \in \mathbb{R}\}$ in general,
    in the context of this question, we are only concerned with the values of $h$ that correspond to possible
    heights of water that can be put into the tower.
    
    Because there cannot be a negative amount of water in the tower, we know that $h \geq 0$.
    Similarly, because the water tower itself is $60m$ tall, the amount of water that can be put in the tower
    is limited to $60m$, meaning that $h \leq 60$.
    Taking these restrictions together: $$0m \leq h \leq 60m$$

    Finally, since $h$ can vary freely within the domain and is not determined by another quantity in the function $g(h)$,

    and since $V$ is determined by the function $g(h)$,
    
    therefore, for the equation $g(h) = 225\pi h$:

    \fbox{\parbox{\textwidth}{
        \begin{itemize}
            \item The domain is $[0, 60]$
            \item The independent variable is $h$
            \item The dependent variable is $V$
        \end{itemize}
        }
    }

    \hfill ANS $\square$
\end{solution}

\newpage
\begin{exercise}
    \textbf{Given:}
    \begin{itemize}
        \item A cylindrical water tower with a radius of $13m$ and a height of $60m$ is filled to a height of $h$.
        \item The volume $V$ of water​ (in cubic​ meters) is given by the function​ $g(h) = 169\pi h$.
    \end{itemize}

    \textbf{Find:}
    \begin{itemize}
        \item The appropriate domain of the function, the independent and dependent variables
    \end{itemize}
\end{exercise}
\rule{\textwidth}{0.4pt}

\begin{solution}
    While the function $g(h) = 169\pi h$ would accept any value from $\{ h \mid h \in \mathbb{R}\}$ in general,
    in the context of this question, we are only concerned with the values of $h$ that correspond to possible
    heights of water that can be put into the tower.
    
    Because there cannot be a negative amount of water in the tower, we know that $h \geq 0$.
    Similarly, because the water tower itself is $60m$ tall, the amount of water that can be put in the tower
    is limited to $60m$, meaning that $h \leq 60$.
    Taking these restrictions together: $$0m \leq h \leq 60m$$

    Finally, since $h$ can vary freely within the domain and is not determined by another quantity in the function $g(h)$,

    and since $V$ is determined by the function $g(h)$,
    
    therefore, for the equation $g(h) = 169\pi h$:
    
    \fbox{\parbox{\textwidth}{
        \begin{itemize}
            \item The domain is $[0, 60]$
            \item The independent variable is $h$
            \item The dependent variable is $V$
        \end{itemize}
        }
    }

    \hfill ANS $\square$
\end{solution}

\begin{homework}
    \newpage
    \begin{exercise}
        \renewcommand{\theequation}{\arabic{equation}}
        \textbf{Given:}
        \begin{itemize}
            \item $f(x) = 3x^2 - 7x + 2$
        \end{itemize}

        \textbf{Find:}
        \begin{itemize}
            \item Simplify the difference quotient $ \frac{f(x+h)-f(x)}{h}$ for $f(x)$.
        \end{itemize}
    \end{exercise}
    \rule{\textwidth}{0.4pt}

    \begin{solution}
        In order to simplify the difference quotient,
        we must first find $f(x+h)$,
        so that we can substitute it in.

        \begin{align}
            f(x) &= 3x^2 - 7x + 2\\
            f(x+h) &= 3(x+h)^2 - 7(x+h) + 2
                &&\text{(by substitution)} \notag\\
            &= 3(x^2 + 2xh + h^2) - 7(x+h) + 2
                &&\text{(by expansion)} \notag\\
            &= 3x^2 + 6xh + 3h^2 - 7x - 7h + 2
                &&\text{(by distribution)} \notag\\
            &= 3x^2 + 6xh + 3h^2 - 7x - 7h + 2
                &&\text{(by distribution)}
        \end{align}

        Now that we have a value for $f(x+h)$,
        we can substitute the value for it into the difference quotient.

        \begin{align}
            q &= \frac{f(x+h)-f(x)}{h} \notag\\
            &= \frac{( 3x^2 + 6xh + 3h^2 - 7x - 7h + 2 ) - (3x^2 - 7x + 2)}{h}\notag
                &&\text{(by substitution)} \notag\\
            &= \frac{ 6xh + 3h^2 - 7h }{h}\notag
                &&\text{(by simplification)} \notag\\
            &= \frac{ h(6x + 3h - 7) }{h}\notag
                &&\text{(by factoring)} \notag\\
            &= 6x + 3h - 7 \notag
                &&\text{(by simplification)} \notag
        \end{align}

        \fbox{\parbox{\textwidth}{
            $\therefore$ The difference quotient for $f(x)$ is $6x + 3h - 7$
            }
        }

        \hfill ANS $\square$
    \end{solution}

\end{homework}
