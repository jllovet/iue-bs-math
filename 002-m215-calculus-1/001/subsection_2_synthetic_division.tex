\subsection{Synthetic Division}
    Synthetic division is used to determine whether a zero of a function
    is a factor of a polynomial.
    Core algorithm:
    \begin{itemize}
        \item Start with the leading coefficient.
        \item Multiply the zero by that coefficient to produce an intermediate term. 
        \item Add that intermediate term to the next coefficient to produce a remainder.
        \item Multiply the zero by that remainder to produce the next intermediate term.
        \item Repeat until the final coefficient is reached.
        \item If the final remainder $= 0$, then the zero is a factor of the polynomial.
        \item If the final remainder $\neq 0$, then the zero is not a factor of the polynomial.
    \end{itemize}
    \newpage
    \inputminted[linenos, breaklines]{python}{001/synthetic_division.py}
    \newpage

\subsection{Graphics}
    In order to be able to work with graphing images, I am using pgfplots. Here is an example.

    \begin{tikzpicture}
        \begin{axis}[standard,
            width=10cm,
            xtick={-10, -8, -6, -4, -2, 0, 2, 4, 6, 8, 10},
            ytick={-10, -8, -6, -4, -2, 0, 2, 4, 6, 8, 10},
            samples=1000,
            xlabel={$x$},
            ylabel={$y$},
            xmin=-10,xmax=10,
            ymin=-10,ymax=10
        ]

            \addplot[line width=0.4mm, red, name path=F,domain={-10:10}]{x^2};
        \end{axis}
    \end{tikzpicture}
