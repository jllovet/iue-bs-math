\documentclass{article}

\usepackage{parskip}
\usepackage{graphicx}
\graphicspath{ {./images/} }

\usepackage{fontspec}
\setmainfont{Gentium Plus}
\newfontfamily\greekfont[Script=Greek]{Gentium Plus}
\newfontfamily\greekfontsf[Script=Greek]{Gentium Plus}

\usepackage{polyglossia}
\usepackage{csquotes}
\setdefaultlanguage{english}
\setotherlanguage[variant=ancient]{greek}

\usepackage[backend=biber, bibencoding=utf8]{biblatex}
\addbibresource{bibliography.bib}

\title{A Sample Article Illustrating the Use of Bibliographies, UTF8, and Embedded Images}
\author{Jonathan Llovet}
\date{}

\begin{document}
\maketitle

I am going to be reading a Calculus textbook in school by \textcite{briggsCalculusEarlyTranscendentals2019}.

Motivated by my coming courses in Calculus,
recently in my research at the intersection of mathematics and philosophy,
I have been studying the infinite.

In particular, I'm reading \textit{Everything and More} by \textcite{wallaceEverythingMoreCompact2010}
and \textit{Contributions to the Founding of the Theory of Transfinite Numbers}
(\textcite{cantorContributionsFoundingTheory1955}, \textcite{cantorGesammelteAbhandlungenMathematischen2013}).
Additionally, in a study group with one of my friends,
I am revisiting Leibniz's \textit{Monadologie} (\textcite{leibnizMonadologieFranzosischDeutsch2017}).

To illustate that polytonic Ancient Greek is supported well, here are the opening lines of the Iliad. \textcite{homereHomeriOpera1920}

\begin{quote}
    \begin{greek}[variant=ancient]
        μῆνιν ἄειδε θεὰ Πηληϊάδεω Ἀχιλῆος\\
        οὐλομένην, ἣ μυρί᾽ Ἀχαιοῖς ἄλγε᾽ ἔθηκε,\\
        πολλὰς δ᾽ ἰφθίμους ψυχὰς Ἄϊδι προΐαψεν\\
        ἡρώων, αὐτοὺς δὲ ἑλώρια τεῦχε κύνεσσιν \hfill 5\\
        οἰωνοῖσί τε πᾶσι, Διὸς δ᾽ ἐτελείετο βουλή,\\
        ἐξ οὗ δὴ τὰ πρῶτα διαστήτην ἐρίσαντε\\
        Ἀτρεΐδης τε ἄναξ ἀνδρῶν καὶ δῖος Ἀχιλλεύς.\\
    \end{greek}
\end{quote}

Here is an equation.

\begin{equation}
    f(x) = \sum_{n = 1}^{\infty} \frac{a}{x} + 2x - 1
\end{equation}

\newpage
\begin{figure}[t]
    \includegraphics[width=\textwidth]{example_flow.pdf}
    \centering
    \caption{System diagram generated with the Python package \textit{diagrams} based on an example from \textcite{senCreateNeatTechnical2022}.}
    \label{fig:system1}
\end{figure}

The \textit{diagrams} package is available on Github.
It is an extension of the graphing tool graphviz, which creates diagrams from plain-text definitions.

\begin{quote}
    Diagrams lets you draw the cloud system architecture in Python code.
    It was born for prototyping a new system architecture without any design tools. You can also describe or visualize the existing system architecture as well.
    Diagram as Code allows you to track the architecture diagram changes in any version control system.

    \textcite{DiagramsDiagramCode}
\end{quote}

\begin{figure}[h]
    \includegraphics[width=\textwidth]{clusters.pdf}
    \centering
    \caption{Another diagram that shows the use of clusters in the \textit{diagrams} package. \textcite{ClustersDiagrams}}
    \label{fig:clusters}
\end{figure}

\newpage
\printbibliography

\end{document}
