\documentclass[leqno]{article}

\usepackage{parskip}
\usepackage{graphicx}
\graphicspath{ {./images/} }

\usepackage{fontspec}
\setmainfont{Gentium Plus}
\newfontfamily\greekfont[Script=Greek]{Gentium Plus}
\newfontfamily\greekfontsf[Script=Greek]{Gentium Plus}

\usepackage{polyglossia}
\usepackage{csquotes}
\setdefaultlanguage{english}
\setotherlanguage[variant=ancient]{greek}

\usepackage[backend=biber, bibencoding=utf8]{biblatex}
\addbibresource{bibliography.bib}

\usepackage{amsmath}
\usepackage{amsfonts}

\title{Notes on Fisher and Ziebur's\\\textit{Integrated Algebra and Trigonometry}}
\author{Jonathan Llovet}
\date{\today}

\begin{document}
\maketitle

I am going to be reading \textit{Calculus, Early Transcendentals} by \textcite{briggsCalculusEarlyTranscendentals2019} in school.

To prepare to read \textcite{briggsCalculusEarlyTranscendentals2019},
here I am reviewing algebra and trigonometry,
to help me identify gaps in my knowledge and
to help me practice presenting my work in \LaTeX.

\newpage
\section{Number Systems - The Integers}

To get things going with their presentation of algebra,
\textcite{fisherIntegratedAlgebraTrigonometry1962}
introduce the system of real numbers.
In their discussion, they explicitly presume a working familiarity
with the rules of arithmetic and basic number systems.
The first set of numbers they discuss are the positive integers.

\subsection{Laws Applicable to Positive Integers}

\begin{align}
    &(a+b)+c = a+(b+c) &&\text{Associative Law of Addition}\\
    &(ab)c = a(bc) &&\text{Associative Law of Multiplication}\\
    &a+b = b+a &&\text{Commutative Law of Addition}\\
    &ab = ba &&\text{Commutative Law of Multiplication}\\
    &a(b+c) = ab+ac &&\text{Distributive Law}\label{E:DistributiveLaw}\\
    &(b+c)a = ba+ca &&\text{Distributive Law by Comm. Law of Mult.}\label{E:DistributiveLawByCommLawofMult}
\end{align}

In addition to these, Fisher and Ziebur point out the following interesting pair of properties.
For any two positive integers \(a, b \)
\begin{align}
    &\text{there is another integer \(c\) such that} &&a+b = c\\
    &\text{there is not always an integer \(d\) such that} &&a-b = d
\end{align}

Rephrasing these in terms of sets:
\begin{align}
    &\forall \, a, b \, \{a, b \in \mathbb{Z} \,|\, a+b \in \mathbb{Z}\}\\
    &\exists \, a, b \, \{a, b \in \mathbb{Z} \,|\, a-b \not\in \mathbb{Z}\}
\end{align}

\subsection{Extending the Integers}

To deal with this limitation
where we can't perform subtraction on all pairs of integers,
we can introduce the negative integers and zero
to produce the set of all integers.

This is fine and good, but we haven't fixed all our issues yet.
Even with the integers extended to include negative integers and zero,
we are still not able to perform division for every pair of integers.

\begin{align}
    &\exists \, a, b \, \{a, b \in \mathbb{Z} \,|\, a/b \not\in \mathbb{Z}\}
\end{align}

\newpage
Begin interpolation.

As an aside, \textcite{fisherIntegratedAlgebraTrigonometry1962} claim the following:
\begin{quote}
    It can be shown that the rules of arithmetic
    for integeres are the only ones that preserve
    the associative, commutative, and distributive
    laws.
\end{quote}

I'm not convinced of this just yet, but I suspect that some inquiries into abstract algebra will help me see whether this is the case.

End interpolation.

\newpage
\printbibliography

\end{document}
