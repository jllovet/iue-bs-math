\documentclass{article}

\usepackage{parskip}
\usepackage{graphicx}
\graphicspath{ {./images/} }

\usepackage{fontspec}
\setmainfont{Gentium Plus}
\newfontfamily\greekfont[Script=Greek]{Gentium Plus}
\newfontfamily\greekfontsf[Script=Greek]{Gentium Plus}

\usepackage{polyglossia}
\usepackage{csquotes}
\setdefaultlanguage{english}
\setotherlanguage[variant=ancient]{greek}

\usepackage[backend=biber, bibencoding=utf8]{biblatex}
\addbibresource{bibliography.bib}

\title{Notes on Fisher and Ziebur's\\\textit{Integrated Algebra and Trigonometry}}
\author{Jonathan Llovet}
\date{\today}

\begin{document}
\maketitle

I am going to be reading \textit{Calculus, Early Transcendentals} by \textcite{briggsCalculusEarlyTranscendentals2019} in school.

To prepare to read \textcite{briggsCalculusEarlyTranscendentals2019},
here I am reviewing algebra and trigonometry,
to help me identify gaps in my knowledge and
to help me practice presenting my work in \LaTeX.

\newpage
\section{The System of Real Numbers}

To introduce algebra, \textcite{fisherIntegratedAlgebraTrigonometry1962} introduce the system of real numbers.
To do so, they presume a working familiarity
with the rules of arithmetic and basic number systems.
The first set of numbers they discuss are the positive integers.



\newpage
\printbibliography

\end{document}
